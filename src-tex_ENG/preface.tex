Für die Laien in buddhistischen Ländern ist es seit langem
Tradition\linebreak 
jeden Neu- und Vollmond ein örtliches Kloster
zu besuchen, um sich\linebreak 
einen Dhamma-Vortrag anzuhören. 
In der Tat ermutigte der Buddha selbst seine Sangha
diese vierzehntägige Praxis aufrechtzuerhalten. 
Als mir vorgeschlagen wurde, dass ich das Internet nutzen könnte, 
um kurze Dhamma-Reflexionen jeden Neu- und
Vollmond zu versenden, war ich von dieser Idee erst nicht
überzeugt, beschloss aber dann doch es
auszuprobieren.
Obwohl wir in einer Welt leben, in der die Phasen des Mondes nicht mehr viel bedeuten, hilft es nach wie vor vielen an die alte Tradition, von der wir ein Teil sind, erinnert zu werden.

Im September 2007 begannen wir Verse aus dem
Dhammapada zu versenden, die wir aus `A Dhammapada for Contemplation', 2006\linebreak ausgewählt haben. Jeden `Mond-Tag' wurde ein Vers angeboten, unter\-stützt durch
einen kurzen Kommentar darüber.
Dieses Programm ist
mittlerweile durch Mundpropaganda und weitergeleitete
Emails recht bekannt geworden. 
Ich höre von Menschen
aus verschiedenen Teilen der Welt, dass sie es schätzen,
von Zeit zu Zeit eine Erinnerung an eine\linebreak alte Lebensart zu
erhalten, wenn sie ihrem geschäftigen Leben nach\-gehen.
Andere freuen sich jeden Neu- und Vollmond darauf ihre Emails zu öffnen, wenn sie am Abend von der Arbeit heimkommen. 
Diese Dhamma-Reflexionen werden
privat genutzt, vielfach kopiert, übersetzt und herum\-gereicht. Ich habe auch gehört, dass sie die Grund\-lage für\linebreak Diskussionen bei einigen wöchentlichen Treffen von Meditations\-gruppen bilden.

Es ist meine Absicht gewesen, dass, in dem ich meine persönlichen Reflexionen auf diesem Wege mit anderen teile, sie sich ermutigt fühlen mögen, ihre eigenen kontemplativen Fähigkeiten zu nutzen.
Es scheint bei buddhistisch Praktizierenden im Westen eine Tendenz zu geben,\linebreak Frieden und Verständnis durch das
Stoppen des Denkens zu finden. Doch der Buddha spricht
davon, dass wir durch ’yoniso manasikara’ oder weises Betrachten, die wahre Natur unseres Geistes zu sehen bekommen, nicht einfach durch das Stoppen des Denkens.

Ich möchte mich bei einigen bedanken, die bei der
Vorbereitung dieses Materials geholfen haben. 
Für die Dhammapada Verse habe ich verschiedene verlässliche
Übersetzungen herangezogen. 
Insbesondere\linebreak habe ich die
Arbeit von Ven. Narada Thera (BMS 1978), Ven. Ananda Maitreya Thera (Lotsawa 1988), Daw Tin Mya und die Herausgeber der burmesischen Pitaka Vereinigung (1987)
und Ajahn Thanissaro verwendet. 
Für die (kommentariell) überlieferten Geschichten zu den Versen\linebreak habe ich auch die
Internet-Seite www.tipitaka.net als Quelle genutzt.
Als ich von genügend Leuten gehört habe, dass eine
Buch-Version von diesen Reflexionen nützlich wäre, habe
ich mich an meinen guten Freund Ron Lumsden gewandt.
Sein beachtliches Geschick für redaktionelle\linebreak Bearbeitung
hat dazu beigetragen, meine Arbeit für ein größeres Publikum zugänglich zu machen.

Möge das Gute, das aus der Zusammenstellung dieses kleinen Büchleins entsteht, mit allen, die an der Produktion und dem Sponsoring\linebreak beteiligt waren, geteilt werden.  Mögen alle, die den Weg suchen ihn\linebreak finden und die Freiheit am Ende des Weges erleben.  Mögen alle Wesen den Weg suchen.



%\bigskip
{\par\raggedleft
Bhikkhu Munindo\\
Aruna Ratanagiri, Regenzeitretreat 2009
%Northumberland\\
%Vereinigtes Königreich\\
\par}

