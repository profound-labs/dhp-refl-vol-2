
A radiância dos ensinamentos do Buddha chega-nos até hoje, aqui e agora, iluminando o caminho que nos leva em direcção à libertação do sofrimento. A luminosidade do Dhamma é reflectida nas palavras e acções daqueles que seguem os ensinamentos do Buddha. Milhões de homens e mulheres tiveram pelo menos um vislumbre desta luz entre a escuridão do nosso mundo. Guiados pelas palavras do Professor, esses seres reconheceram o quão calorosa e permeada de clareza é a sua verdadeira natureza.

Este livro contém uma selecção de 52 versos retirados do Dhammapada. Acompanhando cada verso encontra-se um curto parágrafo de Ajahn Munindo, um monge Budista Theravāda, presentemente abade do mosteiro de Aruna Ratanagiri, em Northumberland, Reino Unido. O Dhamapada contém na sua totalidade 423 versos, sendo cada um deles um exemplo intemporal da radiante sabedoria do Buddha. Eles são «…artefactos muito antigos que, miraculosamente, contêm em si as profundas realidades professadas pelo Buddha», tal como Thomas Jones descreve no epílogo de ‘A Dhammapada for Contemplation’ 2006, publicação da qual estes versos\linebreak foram retirados.

A mensagem dos versos originais e dos comentários que Ajahn Munindo oferece não são recebidos nem sob a forma de sermão nem de doutrina. Os versos são traduções de uma linguagem oriental que é, pela sua própria natureza, culturalmente indirecta. O formato dos versos é curioso e interessante, apresentando características do \textit{haiku} japonês ou das estrofes do I Ching. Eles induzem à compreensão do leitor. Por vezes a sequência das linhas num verso surgem invertidas. Existe uma noção de que os seus começos deveriam vir no final pois os exemplos aparecem antes do pensamento raiz do verso ser expresso, tal como podemos ver no exemplo do verso 377: «\textit{Assim como o jasmim larga as suas flores velhas, deixai também vós, ó bhikkhus, o desejo e o ódio caírem por terra.}» Estes estilo de indução é mais suave do que aqueles aos quais os leitores ingleses estão habituados. Aqui somos gentilmente conduzidos ao pensamento principal do verso: «\textit{… deixai também vós, ó bhikkhus, o desejo e o ódio caírem por terra.}». Em geral, as linguagens ocidentais usam um estilo dedutivo, no qual o leitor é mais firmemente levado à compreensão. Podemos observar isso no verso 377 se lhe retirarmos o tema e o posicionarmos no início: «\textit{Deixai também vós, ó bhikkhus, o desejo e o ódio caírem por terra, tal jasmim que deixa cair as suas flores velhas.}»

Os versos do Dhammapada são estéticos e indirectamente instrutivos: «\textit{Abram mão daquilo que está por vir, abram mão daquilo que já passou e abram mão daquilo que está entre ambos. Para o coração liberto não mais haverá morte e nascimento.}» v.348. O fim do percurso (não mais haverá morte e nascimento.) vem no fim do verso. Esta estrutura indutiva possibilita naturalmente o leitor a reflectir na ‘chegada’, tendo de abrir mão de tudo.

Os comentários de Ajahn Munindo são apresentados na mesma forma indutiva; eles convidam à participação. Instrutores espirituais de todas as linhagens podem ter a tendência de falar demasiado acerca da possibilidade da jornada ao invés de se juntarem a nós enquanto caminhamos no caminho. Os leitores deste livro têm a oportunidade de ver como ser parte de uma investigação partilhada pode naturalmente suscitar uma nova compreensão. O processo apresenta-nos uma nova parte de nós próprios. O Buddha deixou bastante claro que tínhamos de ser nós a fazer o esforço: «\textit{Não posso fazer mais que apontar o caminho.}» ‘Dhammapada – Reflexões’ ajuda-nos a realizar esse esforço. No seu prefácio Ajahn Munindo identifica a reflexão sábia (\textit{yoniso manasikāra}) como um elemento primário do caminho espiritual. Os seus comentários plantam as sementes para a reflexão. Cabe-nos a nós nutrir essas sementes e observar o seu crescimento.

Talvez decida deixar este pequeno livro aberto na sua mesa de oração com um novo verso para ponderar em cada semana ou talvez o leve consigo quando viajar. Como alguém que tem vindo a estudar estes versos há já vários meses, estou certo de que os achará inspiradores. Eles são algo maravilhoso: a sua gentil natureza instrutiva e habilidosa guia-nos. Passo a passo abordamos o Dhamma, o qual tem o poder de realçar todos os aspectos das nossas vidas. 

Em nome de todos os recipientes destas mensagens Dhammasakaccha periódicas, gostaria de expressar a minha gratidão a Ajahn Munindo pelo tempo e energia que disponibilizou na preparação do material para esta publicação. Aquilo que já se encontrava disponível, em correio electrónico, para algumas centenas de indivíduos pode ser agora lido e partilhado sob o formato deste bonito livro.

\bigskip
{\par\raggedleft
Ron Lumsden,\\
Little Oakley, Essex. 2009
\par}

