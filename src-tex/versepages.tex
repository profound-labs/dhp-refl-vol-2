
%% == 1 ==

\begin{dhpVerse}{85}
\label{dhp-85}
Few are those who reach the beyond.\\
Most pace endlessly back and forth,\\
not daring to risk the journey.
\end{dhpVerse}

\begin{dhpRefl}
To imagine anything truly new is not really possible. Most of what we think is a rearrangement of the past. If our goal is `the beyond,' freedom from suffering, we need a distinctly new approach. The predictability of what we find familiar can make us feel safe, even if it is tedious. This is the endlessly pacing back and forth. And this is how we live with clinging. What would it be like to not-cling; to trust in here-and-now awareness, informed by Dhamma? It takes courage to let go of the familiar and open up to the unknown. Our commitment to the precepts and our training in mindfulness are our protectors. With these in place we can dare to take this new journey.
\end{dhpRefl}

%% == 2 ==

\begin{dhpVerse}{194}
\label{dhp-194}
Blessed is the arising of a Buddha;\\
blessed is the revealing of the Dhamma;\\
blessed is the concord of the Sangha;\\
delightful is harmonious communion.
\end{dhpVerse}

\begin{dhpRefl}
Harmony is a pleasing resonance born of diverse elements: notes struck on an instrument; colours combined in an image; views shared in a dialogue. Diversity is not in itself an obstruction to harmony and concord. Indeed, contrast can bring a richness and depth to experience. Living mindfully, we learn where the potential for beauty lies. Imbalance too can be beautiful if there is harmony; wisdom, skill and sensitivity come together and beauty is manifest. This is a rare blessing.
\end{dhpRefl}

%% == 3 ==

\begin{dhpVerse}{202}
\label{dhp-202}
There is no fire like lust,\\
no distress like hatred,\\
no pain like the burden of attachment,\\
no joy like the peace of liberation.
\end{dhpVerse}

\begin{dhpRefl}
How can we stay focused on the path that leads to clarity and unshakeable peace? Greed, aversion and delusion distort our thinking. Lust can appear attractive; we are pulled into its vortex. Hatred can appear attractive; we feel compelled to do harm. Attachment is rooted in the false belief that clinging makes us happier. The truth is that lust burns, hatred obstructs intelligence and attachments spoil that which is beautiful in life. We might begin to see this truth for ourselves when we train our attention to see through the outer appearance of things. A deeper understanding of these three poisons encourages us to hold back from following their impulses. This is the understanding that leads to liberation.
\end{dhpRefl}

%% == 4 ==

\begin{dhpVerse}{302}
\label{dhp-302}
It is hard to live the life of renunciation;\\
its challenges are difficult to find pleasant.\\
Yet it is also hard to live the householder's life;\\
there is pain when associating with those\\
among whom one feels no companionship.\\
To wander uncommitted is always going to be difficult;\\
why not renounce the deluded pursuit of pain?
\end{dhpVerse}

\begin{dhpRefl}
The Buddha uttered this verse to a monk who had been indulging in deluded feelings of self-pity: `Surely nobody is having as hard a time as I am.' Fortunately for him, this monk received a wise reflection in just the right way, at just the right time, so that he could see what he was doing and let go. When we are not attentive in the present moment, we tend to blame our misfortune outwardly, or we blame ourselves inwardly. Either way we increase the pain by forgetting to expand awareness and fully accommodate the suffering. Suffering is the right response to our resisting reality. If we don't cling, we don't suffer. Suffering is the message. It is not something going wrong. We don't have to get rid of suffering; we need to listen to it.
\end{dhpRefl}

%% == 5 ==

\begin{dhpVerse}{111}
\label{dhp-111}
A single day lived\\
with conscious intention and wisdom\\
is of greater value than a hundred years\\
lived devoid of discipline and manifest wisdom.
\end{dhpVerse}

\begin{dhpRefl}
The best offering we can make to the Buddha is to live wisely. We all know the consequences of living in accordance with preferences: we feel divided, not whole. When conditions conspire to be agreeable we lose ourselves in the happiness we have gained; when conditions become disagreeable we despair over what we have lost. Wisdom `sees' both gain and loss - wisdom sustains the awareness which makes us free.
\end{dhpRefl}

%% == 6 ==

\begin{dhpVerse}{24}
\label{dhp-24}
Those who are energetically\\
committed to the Way,\\
who are pure and considerate in effort,\\
composed and virtuous in conduct,\\
steadily increase in radiance.
\end{dhpVerse}

\begin{dhpRefl}
The Buddha's path to radiance is about letting go. We are encouraged to consider things carefully and consistently until we see into their true nature; uncertainty, instability, impermanence. We train ourselves to let go with an understanding born of investigation, not out of rejection or denial. There is a deluded path of practice which leads to radiance arising not from letting go, but from clinging to views. Holding fast to ideas and feelings of righteousness can bring about intense feelings of self-importance. There may be energy, enthusiasm, conviction, but this path leads to division and disharmony. If our effort is at one with reality there will be inner contentment, radiating well-being.
\end{dhpRefl}

%% == 7 ==

\begin{dhpVerse}{384}
\label{dhp-384}
All chains of confinement fall away\\
from those who see clearly\\
and know well the states\\
of concentration and insight.
\end{dhpVerse}

\begin{dhpRefl}
  A visitor to the monastery asked the meditation master Venerable Ajahn Chah how we can practise concentration meditation (\emph{samādhi}) when in reality there is no self. The teacher explained that when we are developing concentration we work with a self. When we are developing insight (\emph{vipassanā}) we work with non-self. Then when we truly know what's what, we are beyond both self and non-self.
\end{dhpRefl}

%% == 8 ==

\begin{dhpVerse}{201}
\label{dhp-201}
Victory leads to hatred,\\
for the defeated suffer.\\
The peaceful live happily,\\
beyond victory and defeat.
\end{dhpVerse}

\begin{dhpRefl}
Those who live beyond victory and defeat are called `the peaceful', but not because they are devoid of feelings. They are not `beyond' because they have escaped all feelings of pain and loss, but because they have escaped the confidence trick of self. Self is like a rainbow. From a distance it appears real and substantial; as you get closer it appears less solid. If we hold too tightly to our sense of self, we get lost in views about what makes us happy. We believe that winning is all that matters, not seeing that in the process we cause suffering to others. If we hold too loosely to our sense of self we get lost, this time from a lack of boundaries, becoming overly sensitive and lacking in confidence. Self-respect and self-confidence are the natural consequences of a life lived with integrity and understanding.
\end{dhpRefl}

%% == 9 ==

\begin{dhpVerse}{279}
\label{dhp-279}
`All realities are devoid of an abiding self';\\
when we see this with insight\\
we will tire of this life of suffering.\\
This is the Way to purification.
\end{dhpVerse}

\begin{dhpRefl}
When we are feeling down we might look for some `thing' to pick ourselves up with: an ice cream, a movie, a memory, a book. Dhamma teaches us to go in the opposite direction: to be truly happy we need less, not more. What we need is to let go of `me' and `mine'. Contrary to the popular belief in the supremacy of `self', Dhamma says that nothing can give this `I' contentment. The idea that this `I' and its desires will ever be satisfied is a false assumption. Which of my `selves' has turned out to be reliable and lasting: the happy me, the hopeless me, the serious me, the sloppy me, the wise me, the foolish me? This Teaching says that insight into the nature of `self' unburdens us of this false belief, dispels the myth of self-importance and leads us to purification.
\end{dhpRefl}

%% == 10 ==

\begin{dhpVerse}{63}
\label{dhp-63}
The fool who knows he is a fool\\
is at least a little wise;\\
the fool who thinks that he is wise\\
is assuredly a fool.
\end{dhpVerse}

\begin{dhpRefl}
It is liberating to be able to honestly acknowledge our faults. When we stop pretending to others, we can stop hiding from ourselves. Wisdom recognizes this. Wisdom is the light - to the degree that we are open and honest it can shine through us, and we become transparent to ourselves and others. The light cannot shine through if we take ourselves too seriously, and darkness produces fear. The heart contracts when we feel afraid, producing a distinct sense of `me': `I am a solid somebody'. At one level this self-sense might even feel good. But this `somebody' goes around colliding with other `somebodys', leaving trouble and hurt in its wake. Understanding this can be the birth of genuine humility.
\end{dhpRefl}

%% == 11 ==

\begin{dhpVerse}{300}
\label{dhp-300}
Disciples of the Buddha\\
are fully awake\\
both day and night,\\
taking delight in harmlessness.
\end{dhpVerse}

\begin{dhpRefl}
The state of harmlessness may not immediately strike us as such a great attainment. Imagine, however, what our world would be like if everyone was harmless. Imagine what our own hearts would feel like if all our impulses were harmless. The Buddha's enlightened disciples are already fully awake. Their natural disposition is to abide in a state where only thoughts of harmlessness arise - the intention to generate suffering never takes hold. The noble struggle of those still seeking the way requires a rigorous, honest acknowledgement of ill-will, resentment or bitterness whenever they appear. Rejecting them only compounds the struggle. Gradually our practice equips us with the skill to meet such impulses, harmlessly. Harmlessness is also compassion.
\end{dhpRefl}

%% == 12 ==

\begin{dhpVerse}{392}
\label{dhp-392}
Devotion and respect\\
should be offered to those\\
who have shown us the Way.
\end{dhpVerse}

\begin{dhpRefl}
When we show respect to those who are worthy, that within us which is worthy grows strong. When we express devotion to teachers who have shown us the way, our commitment to that way deepens. The story told by greed and aversion is that giving means losing. All we stand to lose, however, is the stranglehold of `me' and `my way'. What could be gained is gladness and strength of heart.
\end{dhpRefl}

%% == 13 ==

\begin{dhpVerse}{160}
\label{dhp-160}
Truly it is ourselves that we depend upon;\\
how could we really depend upon another?\\
When we reach the state of self-reliance\\
we find a rare refuge.
\end{dhpVerse}

\begin{dhpRefl}
By pointing out that we ourselves are our only true refuge, the Buddha is showing us that inwards is the direction we need to look if we want real security. We pay due care to outer activity, but we can't afford to lose ourselves in it. To be freed from perpetual disappointment in life, we need to know ourselves, fully. When we really know ourselves, we can forget ourselves. Having been released from the painful prison of selfishness, our liberated hearts and minds will be available to serve reality truly in each moment. Generosity, kindness and empathy all manifest naturally when our being is one with truth.
\end{dhpRefl}

%% == 14 ==

\begin{dhpVerse}{211}
\label{dhp-211}
Beware of the attachment that springs from fondness,\\
for separation from those one holds dear is painful,\\
while if you take sides neither for nor against fondness,\\
there will be no bondage.
\end{dhpVerse}

\begin{dhpRefl}
We may feel fondness, but how free are we in the way we relate to such feelings? Those who have realized Dhamma just feel what they feel and know it in terms of reality: feelings are not ultimate, not something to get lost in. When we attach to feelings we distort them. In a moment of grasping at joy we have a sense of enhanced delight, but we can fail to see the disappointment being stored up for later. Conditions change; the happiness fades, and so does the sense of `me' being happy. We are also generating a momentum of clinging to, and thereby getting lost in, unpleasant feelings. Pleasant or unpleasant, feeling is feeling - cling to one and we cling to both. Contemplating this verse, we might discover that mindfully allowing feelings to come and go doesn't diminish them. In fact, the freedom to feel whatever we feel, without being limited by the impulse to grasp, is the path out of bondage.
\end{dhpRefl}

%% == 15 ==

\begin{dhpVerse}{53}
\label{dhp-53}
As many garlands can be made\\
from a heap of flowers,\\
so too, much that is wholesome can be done\\
during this human existence.
\end{dhpVerse}

\begin{dhpRefl}
As garland-makers sit with their heap of flowers in front of them, so we sit in meditation with the potential of our lives before us. In every moment there is the possibility to do something good, to create something new, something beautiful. So, too, with not-doing; in restraining unwholesome impulses we bring beauty into our world. Whatever we do, however we act, the possibility for doing good is always here - at every moment.
\end{dhpRefl}

%% == 16 ==

\begin{dhpVerse}{199}
\label{dhp-199}
While in the midst\\
of those who are greedy,\\
to dwell free from greed\\
is happiness indeed.
\end{dhpVerse}

\begin{dhpRefl}
We hold values such as honesty, selflessness and generosity to be honourable. However, it can be hard to live our lives aligned with these values when most of those around us are travelling in a different direction. The Buddha knew it could be hard, but he says that to dwell thus is the source of happiness. If we betray ourselves to heedlessness, stop and feel the consequences. Don't be in a hurry to escape. Then imagine how the opposite might feel were we to hold to integrity. Likewise, consider how it feels to focus attention on feelings of gratitude instead of indulging in a sense of lack. Not adding anything to this moment or taking anything from it, we incline towards inner peace and contentment. Establishing ourselves in this awareness, we are less likely to be intimidated by others.
\end{dhpRefl}

%% == 17 ==

\begin{dhpVerse}{68}
\label{dhp-68}
A deed is well done\\
when upon reflection no remorse arises:\\
with joy one harvests its fruits.
\end{dhpVerse}

\begin{dhpRefl}
As mindfulness is strengthened and confidence emerges, we experience a growing sense of being our own authority. Fear of imagined external agents passing judgement on us begins to be seen for what it is - imagination. Our own true heart knows that which is wholesome and that which is not. Wise reflection generates the light of awareness, illuminating the way. When we hear the voice of judgement from our false heart, we receive it and allow it to fade away. Joy remains.
\end{dhpRefl}

%% == 18 ==

\begin{dhpVerse}{67}
\label{dhp-67}
A deed is not well done\\
when upon reflection remorse arises:\\
with tears of sorrow one harvests its fruit.
\end{dhpVerse}

\begin{dhpRefl}
Exercising wise reflection as we travel through the inner landscapes of our lives leads us to understanding. We learn how to live well. The valleys of sadness and remorse teach us. Mindfulness and wise reflection equip us with what is needed to make this journey meaningful. Tried and tested awareness shows us how deeds of the past have left their traces in our hearts. These are the signs which point us in the right direction – the way to here-and-now contentment.
\end{dhpRefl}

%% == 19 ==

\begin{dhpVerse}{421}
\label{dhp-421}
Anyone who lives freed from habits of clinging\\
to past, present or future,\\
attaching to nothing,\\
is a great being.
\end{dhpVerse}

\begin{dhpRefl}
`Attaching to nothing' isn't about not having anything. It is about the way we have things. If we drive a car holding the steering-wheel too tightly, we become tired and don't drive so safely. If the wheel is held too loosely, we likewise run the risk of an accident. This verse is telling us that the past, the present and the future can be held in a way that doesn't lead to suffering. And it is with this clear understanding that we practise letting go, not simply from an ideal that we shouldn't cling.
\end{dhpRefl}

%% == 20 ==

\begin{dhpVerse}{390}
\label{dhp-390}
Suffering subsides to the degree\\
that you are free from the intention to cause harm.\\
There is no real greatness\\
if there is no restraint of anger.
\end{dhpVerse}

\begin{dhpRefl}
Being indignant can give us a feeling of importance, but this importance is full of suffering. After the fire and smoke have subsided, we see the damage caused by unrestrained speech and action. Others have been hurt and we are left alone to endure regret and remorse. Lying in the sun might feel agreeable; sweet drinks and snacks are attractive; but their appearance belies their reality. Because they feel good, that does not make them good. The upthrust of anger feels energizing, but such energy comes at a cost. In following it we build up debts. We may feel we are right in taking a certain position, but we are wrong if we get lost in it. If we are serious about being free from suffering, we will want to become aware of even the slightest intention to cause harm.
\end{dhpRefl}

%% == 21 ==

\begin{dhpVerse}{320}
\label{dhp-320}
Like an elephant in battle\\
withstands arrows,\\
I choose to endure\\
verbal attacks from others.
\end{dhpVerse}

\begin{dhpRefl}
When the going gets tough we are free to make the choice to endure if we wish. Or we could choose to react. Nobody outside of ourselves has the authority to force us in either direction. At times when our untrained, reactive habits flare up, it can certainly feel as if someone or something else is in charge: 'I was taken over by something', which means `I lost perspective'. The Buddha never lost perspective. This does not mean he didn't have to deal with some serious unpleasantness. He did, and he made the choice to endure it rather than react. He was fully aware – fully awake to reality – and he knew he had the authority to make that choice. This helps us to consider that at least potentially, we have it too.
\end{dhpRefl}

%% == 22 ==

\begin{dhpVerse}{6}
\label{dhp-6}
Those who are contentious\\
have forgotten that we all die;\\
for the wise, who reflect on this fact,\\
there are no quarrels.
\end{dhpVerse}

\begin{dhpRefl}
In a moment of wakefulness we recognize how fortunate we are and how readily we take it all for granted. If we lose our health we long for it to return, promising ourselves to value it more in the future. If trust is damaged in a valued friendship, we resolve that if it is healed, never again will we allow it to fall into disregard. Wisdom can arise from regular reflection on what we could lose through heedlessness.
\end{dhpRefl}

%% == 23 ==

\begin{dhpVerse}{380}
\label{dhp-380}
We are our own protection;\\
we are indeed our own secure abiding;\\
how could it be otherwise?\\
So with due care we attend to ourselves.
\end{dhpVerse}

\begin{dhpRefl}
How do we exercise `due care' towards ourselves without becoming selfish? Another expression for `due care' is `right mindfulness'. We train ourselves to watch carefully. When caring turns into clinging, the heart grows cold; `me' and `mine' take over as kindness and discernment fade away. We can trust in the power of mindfulness to reveal this process, as and when it is happening. And so gradually we learn to read our hearts: what does it feel like, in the whole body/mind, when the heart is open, receptive and interested? What does it feel like when the heart is closed, resentful, bitter and afraid? As we carefully feel our way into, around, over and under the many moments of obstruction, life teaches us how to let go. The Teacher is telling us that if we could let go fully, we would feel secure, totally.
\end{dhpRefl}

%% == 24 ==

\begin{dhpVerse}{253}
\label{dhp-253}
Those who always look for\\
the faults of others –\\
their pollutions increase\\
and they are far from freedom.
\end{dhpVerse}

\begin{dhpRefl}
We are far from freedom because we have moved away from the place where freedom lies – in the pure heart. Even if water has become polluted, its essential, pure nature remains; pure water, only with something added. It can help to consider that behind the confusion of fear, greed and resentment, the heart that is clear and at peace is already there. We might think that to find peace we have to get something we lack. In reality we have too much. When, for example, the tendency to criticize arises, what happens if we simply watch it? Don't move! When we don't follow the movement of mind we remain where we are most comfortable, at home, in awareness.
\end{dhpRefl}

%% == 25 ==

\begin{dhpVerse}{135}
\label{dhp-135}
Just as a herdsman\\
drives cattle to pasture,\\
old age and death\\
drive living beings.
\end{dhpVerse}

\begin{dhpRefl}
The Buddha would never say anything with the intention of making us feel depressed. So what is the point of this verse on the inevitability of getting old and dying? The point is reality, Dhamma. As long as we deny reality, our efforts feed habits of delusion - our hearts are sapped of vitality. We have been moving towards death since the day we were born. Opening our hearts and minds to this truth of the way it is does not have to be depressing. Contrary to what we might expect, we are freed from having to lie to ourselves. We already know what we have ahead of us and it takes energy to deny it. The Buddha is telling us that behind the veils of delusion is the deathless. To know this would be to no longer be driven.
\end{dhpRefl}

%% == 26 ==

\begin{dhpVerse}{47}
\label{dhp-47}
As a flash-flood\\
can sweep away a sleeping village,\\
so death can destroy those who only seek\\
the flowers of sensual pleasures.
\end{dhpVerse}

\begin{dhpRefl}
The Buddha enjoyed tranquil bamboo groves and inspiring mountaintops, recommending such places for cultivating meditation. Mahākassapa, one of his great disciples, also described in elegant verse the joy of being in nature. These awakened beings preferred particular conditions, but didn't complain when they found themselves in less agreeable surroundings. Their inner journey wasn't obstructed when their preferences were not met. The Buddha spoke of a way to what he called the deathless state - a state that could be realized if we see, clearly, the nature of our senses - seeing, hearing, smelling, tasting, touching and cognizing. After years of dramatic denial of the sense pleasures, he discovered a middle way between indulgence and avoidance. This way of right mindfulness knows fully when sense pleasure arises, and learns from sense pain too. No fixed position.
\end{dhpRefl}

%% == 27 ==

\begin{dhpVerse}{81}
\label{dhp-81}
As solid rock\\
is unshaken by the wind,\\
so are those with wisdom undisturbed,\\
whether by praise or blame.
\end{dhpVerse}

\begin{dhpRefl}
We are all subject to the forces of praise and blame, no matter who we are or what our calling in life. So how can we live in this world without becoming victim to these forces? The Buddha says that wisdom is the answer. Whatever the external conditions, when wisdom is firmly established our hearts remain unshaken. Wise beings feel the impact of the forces of the world, but don't lose their balance. To arrive at wisdom, we must first understand the pain of losing balance. Remember that freedom from suffering comes from being mindful of it. Feeling hurt by criticism does not mean we have failed; it points to where and when to be mindful.
\end{dhpRefl}

%% == 28 ==

\begin{dhpVerse}{39}
\label{dhp-39}
There is no fear\\
if the heart is uncontaminated by the passions\\
and the mind is free from ill-will.\\
Seeing beyond good and evil,\\
one is awake.
\end{dhpVerse}

\begin{dhpRefl}
Soon after his enlightenment the Buddha was asked if he was a god or a human. He replied that he was simply awake. Being awake, being perfectly aware, he could see and understand everything in terms of reality. His inner vision had been freed from the distortions of craving and resentment and his heart was now perfectly at peace. The contaminations of greed and ill-will were removed and the natural state of clear seeing was manifest. Fear could no longer arise. With all fear dispelled, only wisdom and compassion remain.
\end{dhpRefl}

%% == 29 ==

\begin{dhpVerse}{34}
\label{dhp-34}
Like a fish, which on being dragged \\
from its home in the water \\
and tossed on dry land \\
will thrash about,\\
so will the heart tremble\\
when withdrawing from the current of Māra.
\end{dhpVerse}

\begin{dhpRefl}
Probably we can all relate to this image. It's about how it feels when we attempt to let go of our habits. The current of Māra – a Buddhist term for the manifestation of delusion –  is the force of distraction. Even after years of practice, our addiction to distraction can still return, sometimes subtle, sometimes coarse. The more intense our resolve to let go, the more convincing the obstructions can appear. This is not necessarily something going wrong, it is natural. We are all in this together and it can help to remember we need each other's support on this challenging journey.
\end{dhpRefl}

%% == 30 ==

\begin{dhpVerse}{103}
\label{dhp-103}
One might defeat alone in battle\\
a thousand thousand men,\\
but one who gains self-mastery\\
is by far the greater hero.
\end{dhpVerse}

\begin{dhpRefl}
This verse is not an endorsement of violence of any kind. It is about recognizing that which is truly worthy. When the ego sees itself as a hero there are problems. Dhamma says that none of ego's achievements are particularly valuable. What truly matters is the task of dropping our attachment to, and identification with, the personality structure. If we think we are being spiritual by demonizing ego, we make the task even harder. Self-mastery is not served by merely judging the unhappy experience of limited being. It is about investigating until we come to see the way things actually are. Ego has its place. It is a convention or habit of perception that has a natural function. The mistake we make is to assume that it is who and what we are.
\end{dhpRefl}

%% == 31 ==

\begin{dhpVerse}{26}
\label{dhp-26}
Those who are foolish and confused\\
betray themselves to heedlessness.\\
The wise treasure the awareness they have cultivated\\
as their most precious possession.
\end{dhpVerse}

\begin{dhpRefl}
We all forget ourselves from time to time and become lost in confusion. What matters is how long it takes us to remember. We can't rid ourselves of confusion just because we don't like it. But meditation can help. When we make an effort in formal practice to come back to our object of focus over and over again, we are building a certain kind of strength. Even if we don't see it at the time, valuable potential is growing. Meditation can be boring and can even seem pointless. But energy never disappears. The good effort we make to remember to return to our meditation-object can come back to us in daily life as spontaneous here-and-now awareness. Where we might previously have forgotten ourselves and become lost, we suddenly find ourselves again, alert and mindful of what is happening. Such awareness is indeed a treasure.
\end{dhpRefl}

%% == 32 ==

\begin{dhpVerse}{175}
\label{dhp-175}
Swans fly through the air.\\
Adept yogis transport themselves through space.\\
Wise beings transcend worldly delusion\\
by outwitting the hordes of Māra.
\end{dhpVerse}

\begin{dhpRefl}
As we journey along the spiritual path, we will inevitably hear of magical feats performed by spiritual masters. Even if some people do experience profound states of absorption and have powerful psychic abilities, wanting such experiences to occur can be an obstruction to development. We are wise to remember that Māra can appear in any guise. Without well-established, embodied awareness, we easily ascribe more value to special experiences than they deserve. The feat the Buddha praised was that of transcending delusion.
\end{dhpRefl}

%% == 33 ==

\begin{dhpVerse}{101}
\label{dhp-101}
A single verse of truth\\
which calms the mind\\
is better to hear than a thousand\\
irrelevant verses.
\end{dhpVerse}

\begin{dhpRefl}
It said that quality counts for more than quantity. If we apply this principle to our effort in practice, we won't worry about how long we have been sitting, how many texts we have studied or how many Dhamma talks we have listened to. It is of more value to listen fully to just one minute of a Dhamma talk than to listen partially for many hours. To listen fully means we listen not just with our ears, but also with our hearts. And to do this we let go of trying to understand. Also, we must let go of all cynicism. Wholesome scepticism is an ally in practice, but cynicism is corrosive. Real listening requires trust.
\end{dhpRefl}

%% == 34 ==

\begin{dhpVerse}{258}
\label{dhp-258}
Those who speak much\\
are not necessarily possessed of wisdom.\\
The wise can be seen to be at peace with life\\
and free from all enmity and fear.
\end{dhpVerse}

\begin{dhpRefl}
The Buddha often spoke about `the wise' and the benefits of associating with the wise. But how can we be sure someone is wise? One way is to observe how free they are from ill-will. And we might get a feeling for that by listening to them with our hearts. The same is true for fear. Does this person generate peace or contention? If we listen only with our heads, we could be over-impressed by clever speech. But if we let go of trying to know for sure and quietly trust in mindful attention, our own common sense may guide us.
\end{dhpRefl}

%% == 35 ==

\begin{dhpVerse}{195-196}
\label{dhp-195}\label{dhp-196}
Immeasurable is the benefit\\
obtained from honouring those\\
who are pure and beyond fear.\\
Beings who have found freedom\\
from sorrow and grieving are worthy of honour.
\end{dhpVerse}

\begin{dhpRefl}
It is virtue within us that recognizes virtue in others. When we honour this or that virtue in another person, those same qualities are nourished within ourselves. We set Buddha images up high so that we can look up to them and to that which they symbolize. What a privilege to find confidence in a path of practice which points to sorrow and fear as teachings. Such feelings are not who or what we are - there is much more to us than that. In bowing to our teacher, the Buddha, we learn to bow into life and death and to learn from everything.
\end{dhpRefl}

%% == 36 ==

\begin{dhpVerse}{336}
\label{dhp-336}
As water falls from a lotus leaf\\
so sorrow drops from those\\
who are free of toxic craving.
\end{dhpVerse}

\begin{dhpRefl}
The sorrows of life can convince us they are really important. They seem to demand a huge amount of attention. However, the Buddha teaches us that the most important things are mindfulness and wise reflection. If mindfulness practice is mature, we will be able to observe suffering when it arises without becoming too fascinated by it. We will also be able to reflect wisely on the reality of the moods we have, not just be sensitive to them. They are not ultimate – they come and go. And there is a cause for their arising. Once this cause is recognized, the Buddha says, suffering simply falls away.
\end{dhpRefl}

%% == 37 ==

\begin{dhpVerse}{405}
\label{dhp-405}
Those who have renounced the use of force\\
in relationship to other beings,\\
whether weak or strong,\\
who neither kill nor cause to be killed,\\
can be called great beings.
\end{dhpVerse}

\begin{dhpRefl}
We aspire to live with a heart of kindness and wisdom. And it is to these spiritual powers that we need to turn when we seek a resolution to conflict. On occasions when we don't get our own way, we can feel tempted to use force. At an instinctive level part of us might want to fight, manipulate, even be unkind. For this reason we train to prepare ourselves in advance, so that when the fires of resentment, frustration and disappointment flare up we don't abandon our commitment to reality.
\end{dhpRefl}

%% == 38 ==

\begin{dhpVerse}{4}
\label{dhp-4}
If we thoroughly release ourselves\\
from such thoughts as,\\
`They abused me, mistreated me,\\
molested me, robbed me,'\\
hatred is vanquished.
\end{dhpVerse}

\begin{dhpRefl}
If we have held for a long time to thoughts of ill-will, we can forget how it feels to be happy. When the mind is accustomed to negativity, unhappiness can become the norm. But just as the essential nature of water is pure and clear, so the heart in itself is calm and bright. Painful things do happen in life and memories don't necessarily disappear as we might wish. But adding resentment to memories is adding something extra. It is not an obligation. The release from negative thoughts may not be as difficult as it appears. Honestly acknowledging to ourselves that we do have this choice, to allow memories to be memories - nothing extra – can lead to letting go and moving on.
\end{dhpRefl}

%% == 39 ==

\begin{dhpVerse}{11-12}
\label{dhp-11}\label{dhp-12}
Mistaking the false for the real\\
and the real for the false,\\
one suffers a life of falsity.

But, seeing the false as the false\\
and the real as the real,\\
one lives in the perfectly real.
\end{dhpVerse}

\begin{dhpRefl}
If, with hindsight, we are able to see our conceit, do we appreciate what this awareness has revealed? Or do we despair; do we worry that all our years of practice have come to nothing? In this verse the Buddha tells us we can feel good when we catch ourselves being false. The practice is working!  When we learn to see 'the false as the false', we have the chance to let go and live in the real.
\end{dhpRefl}

%% == 40 ==

\begin{dhpVerse}{35}
\label{dhp-35}
The active mind is difficult to tame,\\
flighty and wandering wherever it wills:\\
taming it is essential,\\
leading to the joy of well-being.
\end{dhpVerse}

\begin{dhpRefl}
If you are practising properly there will be difficulties. Don't feel bad just because you are struggling. This is how it is for everyone who is aware. Remember, it is our heedless habits that cause us trouble, not reality, not Dhamma. So we regularly resolve to align all aspects of our life with Dhamma. Remember too, that the more firm our resolve, the more difficulties we might encounter, so modesty in our ambitions can be a good thing.
\end{dhpRefl}

%% == 41 ==

\begin{dhpVerse}{60}
\label{dhp-60}
The night is long\\
for one who cannot sleep.\\
A journey is long for one who is tired.\\
Ignorant existence is long and tedious\\
for those unaware of Truth.
\end{dhpVerse}

\begin{dhpRefl}
Where do we seek truth? We can look for it in books, or another person may tell us something we've never truly heard before. Or we could ask ourselves what type of question draws attention deeper within, like the question: what matters profoundly in life? Merely gathering impressions about reality is living once removed from life. Truth is not in books or other people. Truth is an unobstructed receptivity to what is, here and now; nothing extra and nothing taken away. Our practice means slowly recognizing and removing the obstructions to clear seeing so that Truth can reveal itself.
\end{dhpRefl}

%% == 42 ==

\begin{dhpVerse}{168}
\label{dhp-168}
Do not show false humility.\\
Stand firmly in relation to your goal.\\
Practice, well-observed,\\
leads to contentment\\
both now and in the future.
\end{dhpVerse}

\begin{dhpRefl}
Genuine humility means that when we need a little wise counsel we are able to receive it. It means that if we lose our way we don't react by resisting the truth that is shown to us. What was it that motivated us to begin this journey? Was it a life-changing experience of which we needed to make some sort of sense? Was it faith that inspired us to seek a realization we trusted could be reached? The goal of clear-seeing challenges us to look even more closely than we have done thus far; instead of trying to be something or somebody we are not, we work to admit more fully to the reality that is ever-present, now. We are more careful with our actions and find contentment in this. False contentment arises from complacency and is completely different from the contentment born of wise reflection. Mindful contentment and true humility lead to insight.
\end{dhpRefl}

%% == 43 ==

\begin{dhpVerse}{54}
\label{dhp-54}
The fragrance of flowers or sandalwood\\
blows only with the prevailing wind,\\
but the fragrance of virtue\\
pervades all directions.
\end{dhpVerse}

\begin{dhpRefl}
The power of newspapers, television and internet over people's lives is obvious. The power of virtue is perhaps less obvious. The powers of integrity, patience, forbearance, restraint are not necessarily dramatic - but if they are real, they are formidable forces. The influence of virtue doesn't manifest in the same way as worldliness. Words spoken from a heart of contentment don't inflame the passions, though they can inspire enthusiasm. A simple act of generosity is never forgotten in the heart of the giver. Goodness comes back to please those who do good, whether they expect it to do so or not. When we forget the power of virtue, we risk falling prey to ego's self-seeking agenda. If we can make the effort to restrain an impulse of unskilful habit even for a second, we generate virtuous energy - and energy is never lost.
\end{dhpRefl}

%% == 44 ==

\begin{dhpVerse}{16}
\label{dhp-16}
When we appreciate fully\\
the benefit of our own pure deeds\\
we are filled with joy;\\
here and hereafter there is a celebration of joy.
\end{dhpVerse}

\begin{dhpRefl}
When we begin to actually let go of grasping – which includes dropping attachment to our precious ideals about not-grasping – what disappears is not our wholesome aspiration, but ignorance of the here-and-now reality. Those who have arrived at a wise relationship with life don't fear they may get lost in joy when it arises. Those who truly meet this moment with awareness are able to tolerate sadness and sorrow if they appear. We learn how to let go as we observe the consequence of our grasping. And we can trust that the integrity which comes from keeping the Precepts will take care of us. If we cease avoiding the truth of this moment, we might find the humility, and therein the strength, to accept ourselves fully.
\end{dhpRefl}

%% == 45 ==

\begin{dhpVerse}{177}
\label{dhp-177}
Those who fail to value generosity\\
do not reach the celestial realms. \\
But the wise rejoice in giving\\
and forever abide in bliss.
\end{dhpVerse}

\begin{dhpRefl}
We may or may not believe in celestial realms, but it is important that we value generosity. When the heart is overshadowed with craving we inevitably think about how to get what we need. If the heart is filled with gratitude we naturally think about ways of being generous. And the magic is that the more generous we are, the more contented we feel. On the other hand, if we are always concerned about getting more, even when we have plenty, we are rarely contented.
\end{dhpRefl}

%% == 46 ==

\begin{dhpVerse}{369}
\label{dhp-369}
Bale out the water from your boat;\\
having cut loose from the defiling passions\\
of lust and hatred,\\
unencumbered, sail on\\
towards liberation.
\end{dhpVerse}

\begin{dhpRefl}
Our heart grows lighter as we see how heavy a burden the untamed passions are. Maybe we have asked ourselves, `What is it that keeps dragging me down?' When we see the connection between the thoughts in our mind and our quality of being, a sense of urgency arises. We won't want to think thoughts that cause us to feel swamped. When we know how to let go of that which leads to obstructions, the vessel upon which we sail towards freedom moves more swiftly.
\end{dhpRefl}

%% == 47 ==

\begin{dhpVerse}{208}
\label{dhp-208}
You should follow the ways\\
of those who are steadfast,\\
discerning, pure and aware,\\
just as the moon follows\\
the path of the stars.
\end{dhpVerse}

\begin{dhpRefl}
Gazing up at the night sky, we might perceive a random scattering of specks of light, or we might recognize an ordering of the stars. It depends on how much we have studied the subject. Whether the apparent chaos of our world defines our experience of life depends on how much we have learned about Truth, Dhamma. There is order within the apparent chaos, but we must look carefully to discern it. The Buddha's very first discourse, `Turning the Wheel of Truth', points directly at that order within chaos that needs to be recognized to make sense of life. This subtle and profound teaching is the Four Noble Truths.
\end{dhpRefl}

%% == 48 ==

\begin{dhpVerse}{90}
\label{dhp-90}
There is no tension\\
for those who have completed their journey\\
and have become free\\
from the distress of bondage.
\end{dhpVerse}

\begin{dhpRefl}
We have a journey to take, and there is a path. The goal is freedom from the experience of being bound by our habits. If we try too hard, grasping at an idea of the goal, attempting to bypass the reality of this moment, we generate stress. If we don't try at all, but constantly lose ourselves in sense stimulus again and again, we generate more stress. To know the right amount of effort requires skill. And it is a skill that we can develop: we lose balance, but instead of habitually judging ourselves, we get interested in what it is we are doing that is creating the problem. Practising with the `right amount' of effort gives the optimal chance of completing the journey.
\end{dhpRefl}

%% == 49 ==

\begin{dhpVerse}{204}
\label{dhp-204}
Health is the greatest gain.\\
Contentment is the greatest wealth.\\
Trustworthiness is the best of kin.\\
Unconditional freedom is the highest bliss.
\end{dhpVerse}

\begin{dhpRefl}
Based on his own experience, the Buddha says that perfect liberation is the ultimate state of ease. He also identifies other sources of well-being. Health in body and mind is a blessing. So too is living with a sense of contentment, not being driven by habits of criticizing and complaining. And trustworthiness: to be someone who is worthy of trust is to feel you have the very best of companions. Trust grows to the degree that we are honest. Trust diminishes to the degree that we are dishonest, with ourselves or others. And the increase or decrease can be incremental. Every small moment of willingness to be true matters. It will have a wholesome result.
\end{dhpRefl}

%% == 50 ==

\begin{dhpVerse}{19}
\label{dhp-19}
Though one may know much about Dhamma,\\
if one does not live accordingly  -\\
like a cowherd who covets another's cattle -\\
one experiences none of the benefits of walking the Way.
\end{dhpVerse}

\begin{dhpRefl}
Though we may own the latest high-speed computer, if we don't learn how to use that computer it is of little value to us. The message of this teaching is that while it matters what we think and believe, what matters more is how we live out the teachings in our actions of body, speech and mind, i.e. how we practise. If we've been educated in how to think, we are fortunate. Our task now is how to develop this ability to think `about' things until it leads us to `know' things. The Buddha didn't want us simply to settle for being able to recite scriptures and work our prayer beads; he wanted us to be able to let go of wrong thinking and really `know' abiding peace. This, surely, must be our goal.
\end{dhpRefl}

%% == 51 ==

\begin{dhpVerse}{112}
\label{dhp-112}
A single day lived\\
with conscious intention and profound effort\\
is of greater value than a hundred years\\
lived in lazy passivity.
\end{dhpVerse}

\begin{dhpRefl}
At times it is inspiration that motivates our practice. At other times it might be despair that energizes us to make an effort. Either way, let us embrace this path of here-and-now awareness with commitment.  If we waver just because we don't get what we want or get what we don't want, we become irresolute. To attend fully to this moment does not mean that we ignore the past or pretend there is no future. It means we don't get lost in the stories our mind creates. We can learn from the past and we can prepare for the future, if our feet are firmly set on the ground of here-and-now. With increasing moments of conscious intention, the depleting tendencies of self-pity diminish. We seek refreshment in the selfless truth of what is.
\end{dhpRefl}

%% == 52 ==

\begin{dhpVerse}{354}
\label{dhp-354}
The gift of Dhamma excels all gifts.\\
The flavour of Dhamma surpasses all flavours.\\
The delight of Dhamma transcends all delights.\\
Freedom from craving is the end of all suffering.
\end{dhpVerse}

\begin{dhpRefl}
This verse is always true, wherever we may be in our practice of Dhamma. For those near the beginning of the path to liberation, it is wonderful to have confidence in the map you hold in your hands. Others further along the way will know the uplift and joy which even small moments of truth may bring. And realized beings who have reached the goal are nourished by the delight of freedom from this burden of suffering. When our habits of getting lost in craving pull us down, the beauty of the undefiled heart is hidden from view; life tastes bland and we forget the many gifts we have received. At those moments, imagine the teacher's smile and gentle reminder, `Begin again, one moment at a time.'
\end{dhpRefl}

